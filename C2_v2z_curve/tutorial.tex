% Issues related to QMC on infinite systems.
\tolerance = 10000
%\documentstyle[aps]{revtex}
%\documentclass[12pt,twoside,fleqn]{article}
%\documentclass[aps,prl,twocolumn,groupedaddress,showpacs]{revtex4}
%\documentclass[aps,prl,groupedaddress]{revtex4}
\documentclass[12pt,aps,prb,groupedaddress,amsmath,amssymb]{revtex4-1}

\usepackage{graphicx}% Include figure files
\usepackage{dcolumn}% Align table columns on decimal point
\usepackage{bm}% bold math
\usepackage{epsf,color}
\usepackage{fancyhdr}
\usepackage{amsfonts}% Needed for \overbar, but did not work anyway
\usepackage{float, rotating, rotfloat}
%\usepackage{capt-of}
\usepackage[section]{placeins} % Prevent figures from moving across sections.

\textwidth 7.5in
\textheight 10.0in

%\voffset -5mm
\topmargin -25mm

\oddsidemargin -11mm
\evensidemargin -11mm

\parindent=0cm
\parskip .15in plus .1in minus .05in
\linespread{1.15}

\pagestyle{fancy}
%\fancyhead[LE,RO]{\slshape \rightmark}
\fancyhead[LE,RO]{\slshape SHCI tutorial}
\fancyhead[LO,RE]{\slshape Umrigar group}
\fancyfoot[C]{\thepage}

%\input vmc.def
%\input dmc.def
%\input projectors.def

\def\erf{{\rm erf}}
\def\erfc{{\rm erfc}}
\def\intr{\int_0^\infty{\rm d}r}
\def\intrinf{\int_0^\infty{\rm d}r}
\def\intrrc{\int_{r_c}^\infty{\rm d}r}
\def\dr{{\rm d}r}

\def\psii{\psi_{i}}
\def\psij{\psi_{j}}
\def\psiij{\psi_{ij}}
\def\psisq{\psi^2}
\def\psironeN{\psi({\bf r}_1,{\bf r}_2,...{\bf r}_N)}
\def\psir1pN{\psi({\bf r}'_1,{\bf r}_2,...{\bf r}_N)}
% \def\psir1jpN{\psi({\bf r}'_{1j},{\bf r}_2,...{\bf r}_N)}
%\def\langle{\left\langle}
%def\rangle{\right\rangle}
\def\ranglep{\rangle_{\psisq}}
\def\Hpsibypsi{{H \psi \over \psi}}
\def\Hpsiibypsi{{H \psii \over \psi}}
\def\HmEpsibypsi{{(H-E) \psi \over \psi}}
\def\HmEpsiibypsi{{(H-E) \psii \over \psi}}
\def\HmEpsijbypsi{{(H-E) \psij \over \psi}}
\def\psiibypsi{{\psii \over \psi}}
\def\psijbypsi{{\psij \over \psi}}
\def\psiijbypsi{{\psiij \over \psi}}
\def\EL{E_{\rm L}}
\def\ELi{E_{{\rm L},i}}

\def\rvec{{\bf r}}
\def\gvec{{\bf g}}
\def\gmax{g_{\rm max}}
\def\rhovec{{\mathbf \rho}}
\def\Lvec{{\bf L}}
\def\Rvec{{\bf R}}
\def\kvec{{\bf k}}
\def\Gvec{{\bf G}}
\def\Gpvec{{\bf G}_p}
\def\kpvec{{\bf k}_p}
\def\Gsvec{{\bf G}_s}
\def\ksvec{{\bf k}_s}
\def\Rsvec{{\bf R}_s}
\def\kpg{\kvec+\Gvec}
\def\kmg{\kvec-\Gvec}
\def\mkpg{-\kvec+\Gvec}
\def\mkmg{-\kvec-\Gvec}
\def\kr{\kvec\cdot\rvec}
\def\Gr{\Gvec\cdot\rvec}
\def\kpgr{(\kpg)\cdot\rvec}
\def\kmgr{(\kmg)\cdot\rvec}
\def\uk{u_{\kvec}}
\def\psik{\psi_{\kvec}}
\def\psimk{\psi_{-\kvec}}
\def\sumg{{\sum_{\Gvec}}}
\def\sumgp{{\sum_{\Gvec^+}}}
\def\egr{e^{i \Gvec\cdot\rvec}}
\def\emgr{e^{-i \Gvec\cdot\rvec}}
\def\ekr{e^{i \kvec\cdot\rvec}}
\def\emkr{e^{-i \kvec\cdot\rvec}}
\def\eppkpgr{e^{i (\kpg)\cdot\rvec}}
\def\eppkmgr{e^{i (\kmg)\cdot\rvec}}
\def\empkpgr{e^{-i (\kpg)\cdot\rvec}}
\def\empkmgr{e^{-i (\kmg)\cdot\rvec}}
\def\epmkpgr{e^{i (\mkpg)\cdot\rvec}}
\def\epmkmgr{e^{i (\mkmg)\cdot\rvec}}
\def\ck{{c_{\kvec}}}
\def\cmk{{c_{-\kvec}}}
\def\ckpg{{c_{\kpg}}}
\def\ckmg{{c_{\kmg}}}
\def\cmkpg{{c_{\mkpg}}}
\def\cmkmg{{c_{\mkmg}}}
\def\csk{{c^*_{\kvec}}}
\def\csmk{{c^*_{-\kvec}}}
\def\cskpg{{c^*_{\kpg}}}
\def\cskmg{{c^*_{\kmg}}}
\def\csmkpg{{c^*_{\mkpg}}}
\def\csmkmg{{c^*_{\mkmg}}}
\def\coskr{\cos{\kr}}
\def\sinkr{\sin{\kr}}
\def\cosGr{\cos{\Gr}}
\def\sinGr{\sin{\Gr}}
\def\coskpgr{\cos{\kpgr}}
\def\sinkpgr{\sin{\kpgr}}
\def\coskmgr{\cos{\kmgr}}
\def\sinkmgr{\sin{\kmgr}}
\def\Ylm{Y_{lm}}
\def\Ylmo{Y_{lm}(\Omega)}
\def\Ylmop{Y_{lm}(\Omega')}
\def\Ylmoone{Y_{lm}(\Omega_1)}
\def\Ylmoonep{Y_{lm}(\Omega_1')}

\def\taubf{{\mathbf \tau}}

\def\Xbar{\overline{X}}
\def\X2bar{\overline{X^2}}

\def\V{{\cal V}}
\def\C{{\cal C}}

% &=& \ck^* \emkr + \sumgp \{\ckpg^*\empkpgr + \ckmg^*\empkmgr\}
% psik+\psimk &=& 2 \Re \left[\ekr + \sumgp \{\ckpg\eppkpgr + \ckmg\eppkmgr\}\right]
%\psik+\psimk &=& 2 \Re \left[\ekr + \sumgp \{\ckpg\eppkpgr + \ckmg\eppkmgr\}\right] \nonumber \\
%&=& 2\left[\Re \ck \coskr -\Im \ck \sinkr
% + \sumgp \Re \ckpg \coskpg - \Im \ckpg \sinkpgr
% + \sumgp \Re \ckmg \coskmg - \Im \ckmg \sinkpmr \nonumber \\
%&=& 2\left[\Re \ck \coskr -\Im \ck \sinkr
% + \sumg \nonumber \\
%&=& 2\left[\coskr\left{\Re \ck + \sumgp\left(\Re(\ckpg+\ckmg)\cosGr - \Im(\ckpg-\ckmg)\sinGr)\right)\right}
%          -\sinkr\left{\Im \ck + \sumgp\left(\Re(\ckpg-\ckmg)\sinGr - \Im(\ckpg+\ckmg)\cosGr)\right)\right}\right]

\def\PsiT{\Psi_T}

\def\beq{\begin{eqnarray}}
\def\eeq{\end{eqnarray}}

\def\cf{{\it cf.}}
\def\eg{{\it e.g.}}
\def\etal{{\it et~al.}}
\def\ibid{\it ibid., \rm}
\def\etc{{\it etc.}}
\def\ie{{\it i.e.}}
\def\viz{{\it viz.}}

\begin{document}

\title{Tutorial on using the Semistochastic Heat-bath Configuration Interaction (SHCI) method}

\author{Adam Holmes$^{a,b}$, Matt Otten$^a$, Junhao Li$^a$ and Cyrus J. Umrigar$^a$}
\affiliation{
$^a$ Theory Center and Laboratory of Atomic and Solid State Physics,\\
Cornell University, Ithaca, NY 14853\\
$^b$ University of Colorado, Boulder, CO}

\maketitle

There are presently two SHCI codes available for quantum chemistry problems-- one in Fortran developed by the Umrigar group
at Cornell (primarily by Adam Holmes, with very efficient parallelisation by Matt Otten), and another in C++
developed by the Sharma group at the University of Colorado.  There is also an SHCI program for treating the
homogeneous electron gas developed in the Umrigar group by Junhao Li.
Today, we will install and use the first program.

Adam Holmes, showed in his talk the potential energy curves for the 12 lowest energy states of C$_2$, computed in a cc-pV5Z basis.
In this tutorial, we will compute a simplified version of this -- we will do the lowest 3 states (you can easily do more if you wish)
in a cc-pVDZ basis.

\begin{enumerate}
\item Download the code by typing:
\begin{verbatim}
git clone https://bitbucket.org/sqmc/hci.git --depth=1
\end{verbatim}
\item
Compile the program:
\begin{verbatim}
cd src
make
\end{verbatim}
\item
\begin{verbatim}
cd ../C2\_v2z_curve
\end{verbatim}
There are 2 input files (for $^1\Sigma_g$ and $^3\Pi_u$ symmetries and integrals files for 9 different geometries.  Take a look at one of these inputs.
The one for $^1\Sigma_g$ computes the 2 lowest states of that symmetry and the one for $^3\Pi_u$ computes just the lowest state.
Run all these inputs by typing:
\begin{verbatim}
./runall
\end{verbatim}
\item Make tables of all these calculations:
\begin{verbatim}
./make_table
\end{verbatim}
\item Plot the energies in the tables:
\begin{verbatim}
gnuplot < i.gnu
evince pes_C2_2z.eps
\end{verbatim}
\end{enumerate}

\end{document}
